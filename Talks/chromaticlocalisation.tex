\documentclass[a4paper]{article} % report for thesis, article for assignments
\usepackage{amsthm} % amsthm
\usepackage[english]{babel}       % Configure hyphenation.
\usepackage[utf8]{inputenc}       % Allow e.g. danish letters.
\usepackage{mathtools}            % Random tools like reverse lim
\usepackage[T1]{fontenc}          % Proper copy-paste and hyphenation of accented characters.
% \usepackage[backend=biber]{biblatex}
% \usepackage[english, science, titlepage]{ku-frontpage/package} % KU front page.

\usepackage{amssymb}              % Load additional symbols, such as \blacksquare
\usepackage{tikz-cd}              % Commutative diagrams.
\usepackage{graphicx}             % Include figures.
\usepackage[margin=3cm,top=3cm, bottom=3cm]{geometry}         % Change margins.
\usepackage{faktor}               % Nice algebraic quotients.
\usepackage{xfrac}                % Nice in-line fraction.
\usepackage{enumerate}            % Easy enumerate
\usepackage{todonotes}
\usepackage{array}
\usepackage{subfiles}
\usepackage{float}
\restylefloat{table}

\usepackage{spectralsequences}

\usepackage{fancyhdr}
\pagestyle{fancy}

\setlength{\emergencystretch}{3em}
\renewcommand{\baselinestretch}{1.1}

%       Theorem environments
%\renewcommand{\qedsymbol}{$\blacksquare$}
\theoremstyle{definition}
\newtheorem{theorem}{Theorem} % Add [section] to also index by section.
\newtheorem{lemma}[theorem]{Lemma}
\newtheorem*{lemmau}{Lemma}
\newtheorem{proposition}[theorem]{Proposition}
\newtheorem{corollary}[theorem]{Corollary}
\newtheorem{definition}[theorem]{Definition}
\newtheorem{example}[theorem]{Example}
\newtheorem{hw}{HW.}
\newtheorem{remark}[theorem]{Remark}
%\theoremstyle{remark}
\newtheorem*{notation}{Notation}
\newenvironment{parts}[1]{\par\noindent\textbf{Part\space#1}}{}
\newenvironment{sketch}{\noindent\textit{Sketch of proof.}}{\hfill$\square$}

%       Characters
\let\epsilon\varepsilon
\let\phi\varphi

%       General math stuff
\newcommand{\toWithMap}[1]{\overset{#1}{\to}}
\newcommand{\toWithMapLong}[1]{\overset{#1}{\longrightarrow}}
\newcommand{\getsWithMap}[1]{\overset{#1}{\gets}}

%       Number systems
\newcommand{\N}{\mathbb{N}}
\newcommand{\Z}{\mathbb{Z}}
\newcommand{\Q}{\mathbb{Q}}
\newcommand{\R}{\mathbb{R}}
\newcommand{\C}{\mathbb{C}}
\newcommand{\F}{\mathbb{F}}
\newcommand{\K}{\mathbb{K}}

%       Spectra
\newcommand{\Sph}{\mathbb{S}}

%       Functional analysis
\newcommand{\Lc}{\mathcal{L}}

%       Brackets
\newcommand{\curly}[1]{ \left\{ #1 \right\} }
\newcommand{\paren}[1]{ \left( #1 \right) }

%       Addition math operators
\DeclareMathOperator{\id}{id}           % Identity function.
\DeclareMathOperator{\im}{im}           % Image of function.
\DeclareMathOperator{\Span}{span}       % Span of X.
\DeclareMathOperator{\Maps}{Maps}       % Span of X.

%       Category theory
\newcommand{\Set}{\mathsf{Set}}         % Sets.
\newcommand{\Top}{\mathsf{Top}}         % Topological spaces.
\newcommand{\Grp}{\mathsf{Grp}}         % Groups.
\newcommand{\Ab}{\mathsf{Ab}}           % Abelian groups.
\newcommand{\Rmod}{\mathsf{_RMod}}      % Left R modules - fixed ring R.
\newcommand{\Modl}[1]{\mathsf{_#1Mod}}  % Left - modules. Parametrized ring.
\newcommand{\Modr}[1]{\mathsf{Mod_#1}}  % Right - modules. Ditto.
\DeclareMathOperator{\coker}{coker}     % Cokernel.
\DeclareMathOperator{\Hom}{Hom}         % Hom set.
\DeclareMathOperator{\Fun}{Fun}         % Function set
\DeclareMathOperator{\End}{End}         % Endomorphisms.
\DeclareMathOperator{\Aut}{Aut}         % Automorphisms.
\DeclareMathOperator{\Spec}{Spec}       % Spectrum of ring.
\DeclareMathOperator{\Ass}{Ass}         % Associated prime ideals.
\DeclareMathOperator{\Ann}{Ann}         % Annihilator.
\DeclareMathOperator{\Ext}{Ext}         % Ext functor.
\DeclareMathOperator{\Tor}{Tor}         % Tor functor.
\DeclareMathOperator{\colim}{colim}     % Colimit of diagram.
\DeclareMathOperator{\Cat}{Cat}         % Small category of categories.

%       Homotopy theory
\DeclareMathOperator{\hofib}{hofib}

%       Set theory
\DeclareMathOperator{\interior}{Int}
\newcommand{\interiorcirc}[1]{\overset{\circ}{#1}}
\newcommand{\boundary}[1]{\delta #1}

%       Functions
\renewcommand{\restriction}[2]{#1_{\mkern 1mu \vrule height 2ex\mkern2mu #2}}

%       Logic
\renewcommand{\iff}{\Leftrightarrow}    % Short iff arrow.
\renewcommand{\implies}{\Rightarrow}    % Left arrow.
\renewcommand{\impliedby}{\Leftarrow}   % Right arrow.

%       Algebra
\newcommand*\quot[2]{{^{\textstyle #1}\Big/_{\textstyle #2}}} % Quotient group
\newcommand*\quotSmall[2]{{^{\textstyle #1}\big/_{\textstyle #2}}} % Quotient group

%       Analysis
\newcommand{\diff}[2]{\frac{d#1}{d#2}}                   % Differentiation d/dx .
\newcommand{\pdiff}[2]{\frac{\partial #1}{\partial #2}}  % Partial d/dx.
\newcommand{\supp}[2]{\sup_{\substack{ #1 } } #2}        % Supremum with substack.
\newcommand{\inff}[2]{\inf_{\substack{ #1 } } #2}        % Infimum with substack.
\newcommand{\limm}[2]{\lim_{\substack{ #1} }  #2}        % Limit with substack.
\newcommand{\norm}[1]{\lVert #1 \rVert}                  % Norm.
\newcommand{\abs}[1]{\lvert#1\rvert}                     % Absolute value.
\newcommand{\Int}[4]{\int_{#1}^{#2}\!{#3}\,d{#4}}        % Integral with proper spacing.

%       Sequences
\newcommand{\seq}[3]{(#1_{#2})_{#2 \in #3}} 				% Sequence (element, index, indexset)
\newcommand{\family}[3]{\{#1_{#2}\}_{#2 \in #3}} 			% Family (element, index, indexset)
\newcommand{\fsets}[2]{#1_1, \ldots, #1_{#2}} 			 	% Finitely many sets (set, numberOfSets)
\newcommand{\funion}[2]{#1_1 \cup \ldots \cup #1_{#2}} 		% Finite union of sets (set, numberOfSets)
\newcommand{\finter}[2]{#1_1 \cap \ldots \cap #1_{#2}}  	% Finite intersection of sets (set, numberOfSets)
\newcommand{\fprod}[2]{#1_1 \times \ldots \times #1_{#2}} 	% Finite product (set, numberOfSets)
\newcommand{\isets}[1]{#1_1, #1_2, \ldots} 			 		% Infinitely many sets (set)
\newcommand{\iprod}[1]{#1_1 \times #1_2 \times \ldots} 		% Infinite product (set)
\newcommand{\ball}[3]{\operatorname{B}_{#1}(#2, #3)}		% Ball (set, center, radius)

%       Article specific
\newcommand{\moduliS}{\mathcal{M}_{FG}^s}
\newcommand{\moduli}{\mathcal{M}_{FG}}
\newcommand{\moduliH}{\widehat{\mathcal{M}}_{FG}}
\newcommand{\Spp}{\mathcal{D}(\mathbb{S}_p)}
%\DeclareMathOperator{Spec}{Spec}

\author{Jon Lindegaard Holmberg}
\title{Recollement and Localisation\\Topics in Algebraic Topology}

\rhead{Jon Lindegaard Holmberg}
\lhead{TopTop 2019/2020}
\chead{Talk ...}

\begin{document}

\maketitle

\tableofcontents

\section{Introduction}
In this article we use recollement in the sense of \cite{barwickglasman} to describe chromatic homotopy theory, and consider the special case of Bousfield localisation of $p$-complete spectra using localisation as Bousfield describes in \cite{bousfield1979localization}.

Section \ref{Section:chromhtpytheory} will give a short introduction to chromatic homotopy theory using \textit{the moduli stack of formal groups} and recollements of $p$-complete spectra. We do not show any proofs, but give a brief exposition. We show two recollement diagrams which leads to the definition of $E(n)$-localisation and $K(n)$-localisation seen in Morava $E$ and $K$-theory. These localisation functors are essential to chromatic homotopy theory. In section \ref{Section:bousfieldlocal} we will use results from \cite{bousfield1979localization}
to describe $E(1)$-localisation of $p$-complete spectra. As it turns out, we can describe such localisation simply by considering the $E(1)$-localisation of the $p$-complete sphere spectrum $\Sph_p$.
\begin{notation}
  We let $\mathcal{D} (\Sph_p)$ denote the stable $\infty$-category of $p$-complete spectra. One could also write this as $\operatorname{Sp}_p$, but we will refrain from that.
\end{notation}
In section \ref{Subsection:recollementpcompletespectra} we will work in the $\infty$-category of stable $\infty$-categories $\Cat_{\text{st}}^\infty$. This is a pointed $\infty$-category by the zero category, but it is \textit{not} stable itself. In this category we will work with \textit{Verdier sequences} which are defined below.

\begin{definition}
  A \textit{Verdier sequence} in the $\infty$-category of stable $\infty$-categories $\Cat_{\text{st}}^\infty$ is a bicartesian square
  \[
    \begin{tikzcd}
      X \arrow{r}{} \arrow{d}{} & Y \arrow{d}{} \\
       0 \arrow{r}{} & Z
    \end{tikzcd}
  \]
  in $\Cat_{\text{st}}^\infty$. Here $Z$ is called the \textit{Verdier quotient}, and can be written as $Y/X$.
\end{definition}

\section{Chromatic homotopy theory}\label{Section:chromhtpytheory}

We start by a brief introduction to chromatic homotopy theory. Chromatic homotopy theory studies stable homotopy theory, and in particular complex oriented cohomology theories such as the complex cobordism theory spectrum $MU$. We study such spectra by means of what is called the \textit{chromatic filtration}. As we will see this chromatic filtration is induced by a height filtration of formal groups. In particular, the localisation $L_{E(n)}$ along the Morava $E$-theories $E(n)$ fits into a chromatic tower
\[
X \toWithMapLong{} \cdots \toWithMapLong{} L_{E(n)}X \toWithMapLong{} L_{E(n-1)}X \toWithMapLong{} \cdots \toWithMapLong{} L_{E(0)}X
\]
where $X$ is a spectrum. These piece can be obtained from localisation along Morava $K$-theories. We will see this in the case where $X$ is a $p$-complete spectrum. Moreover \textit{the chromatic convergence theorem} describes when the homotopy limit over this tower is the $p$-localisation of $X$. We will not consider this theorem.

\subsection{The moduli stack of formal groups}

We start by turning to the the complex cobordism cohomology theory spectrum $MU$, which is universal among such complex oriented cohomology theories. Notice that $MU$ only has homotopy groups in even degrees, and so $\pi_*(MU)$ is a graded commutative ring. If we consider the natural inclusions $$MU \overset{d_0}{\to} MU \otimes MU \overset{d_1}{\leftarrow} MU,$$ we can form a pushout

\begin{center}
\begin{tikzcd}
\operatorname{Spec}(\pi_{*}(MU\otimes MU)) \arrow[rr, "d^1"] \arrow[dd, "d^0 "] &  & \operatorname{Spec}(\pi_{*}(MU)) \arrow[dd] \\ &  & \\
\operatorname{Spec}(\pi_{*}(MU)) \arrow[rr] &  & \mathcal{M}^s_{FG}
\end{tikzcd}
\end{center}
defining $\moduliS$ as a quotient $\Spec(\pi_{*}(MU))/\sim $. This is the \textit{moduli stack of formal groups with strict isomorphisms}. Note that $\moduliS$ is a $\mathbb{G}_m$-torsor with action given by the grading. This is because applying $\Spec$ to the grading
\[
\pi_* MU \to \Z [ t^{\pm 1} ] \otimes_\Z \pi_* MU
\]
yields the action
\[
\mathbb{G}_m \times \Spec (\pi_* MU) \to \Spec (\pi_* MU).
\]
We can quotient out this $\mathbb{G}_m$-action to obtain the stack $\moduli$ which we call the \textit{moduli stack of formal groups}. Now if $X$ is a spectrum, then we have descent data on $\pi_{2*}(MU \otimes_\Sph X)$
\[
\begin{tikzcd}
  d^{1*} \pi_*(MU \otimes_\Sph X) \rar{\tilde{\varepsilon}} \arrow{dr}[swap]{\tilde{d^1}} & d^{0*} \pi_*(MU \otimes_\Sph X) \dar{\tilde{d^0}} \\
  & \pi_*(MU \otimes_\Sph MU \otimes_\Sph X)
\end{tikzcd}
\]
and  $\pi_{2*}(MU \otimes X)$ is a $(\pi_* MU, \pi_*(MU \otimes_\Sph MU))$-module. As we have seen, it corresponds to a quasicoherent $\mathcal{O}_{\moduliS}$-module we denote by $\mathcal{F}_X$.
\iffalse
For a formal group law $f\in R[[x,y]]$ over a $\Z_{(p)}$ algebra $R$ one can define its \emph{height} which is roughly the smallest index where the $p$-series coefficients are non-zero. This notion will lead to a filtration or stratification of $\mathcal{M}^s_{FG}$ by height.
\fi
Consider the formal completion
\[
\moduliH := "\colim_n" \moduli \times_{\Spec (\Z)} \Spec(\Z /p^n)
\]
of $\moduli$ along $p$. This has a stratification of closed sub-ind-stacks
\[
\cdots \subset \moduliH^{\ge 2} \subset \moduliH^{\ge 1}  \subset \moduliH^{\ge 0} = \moduliH
\]
called \textit{the height stratification} of $\moduliH$. As we will see next, we can use the height stratification to define a filtration on the stable $\infty$-category of $p$-complete spectra $\Spp$.

\subsection{Recollements of $p$-complete spectra}\label{Subsection:recollementpcompletespectra}

We can define a filtration $\operatorname{Fil}^h \Spp$ of $\Spp$, and each such filtration step fits into two recollement diagrams. These recollements define the fundamental objects of what is called Morava $E$ and $K$ theory, namely those of $E(n)$ and $K(n)$-local objects for a natural number $n$.
\begin{definition}
  Let $\Spp^{\ge h} \subset \Spp$
  be the full subcategory spanned by all $p$-complete spectra $X$ such that the associated quasi-coherent sheaves $\mathcal{F}_X$ and $\mathcal{F}_{\Sigma X}$ are supported on $\moduliH^{\ge h}$, and let $i_\vee$ denote the inclusion $\Spp^{\ge h} \hookrightarrow \Spp$.
\end{definition}
Now set $\operatorname{Fil}^h \Spp = \Spp^{\ge h}$ for each $h \ge 0$. This defines a filtration on $\Spp$. We have a Verdier sequence
\[
\Spp^{\ge h+1} \overset{i_\vee}{\hookrightarrow} \Spp \overset{j^*}{\to} \Spp^{\le h}.
\]
where we denote the Verdier quotient $\Spp/\Spp^{\ge h+1}$ by $\Spp^{\le h}$. The functor $j^*$ admits a right adjoint $j_*$, and $j_*j^*$ is smashing (see e.g. \cite{luriechromatichomotopytheorynotes}, lecture 23, Theorem 4), so $j_*$ admits a right adjoint by the Smashing Conjecture. Applying the Recollement results we have seen earlier we get a diagram
\iffalse
\begin{center}
  \begin{tikzcd}
    X \arrow[dr, shift left=1ex, "i_\vee"] & & \\
    & X \arrow{r} \arrow[ul, shift right=1ex, "i_\vee"] & X \\
    X \arrow{ur} & &
  \end{tikzcd}
\end{center}
\fi

\begin{equation}
  \label{SppRecollement1}
  \begin{tikzcd}[sep = large]
  \Spp^{\ge h+1} \arrow[dr, shift left=1ex, "i_\vee"] \arrow[dd, shift right=1.5ex, "i^\wedge i_\vee"'] \\
  {} & \Spp  \arrow[dl,shift  right=1ex, "i^\wedge" description]  \arrow[ul, shift left=1ex, "i^\vee"]  \arrow[r, bend left, "j^*"] \arrow[r, above, bend right, "j^\times"] & \arrow{l}{}[swap]{j_*} \Spp^{\le h} \\
  \Spp/ \Spp^{\le h} \arrow[ur, shift right=1ex, "i_\wedge" description] \arrow[uu, shift right=1.5ex, "\sim", "i^\vee i_\wedge"']
  \end{tikzcd}
\end{equation}
where
\[
\Spp^{\le h} \toWithMapLong{j_*} \Spp \toWithMapLong{i^\wedge} \Spp / \Spp^{\le h}.
\]
is a Verdier sequence. The categories present in this diagram are important in chromatic homotopy theory, and so the objects are given specific names.
\begin{definition}
In the recollement diagram (\ref{SppRecollement1}) we say that objects of $\Spp^{\le h}$ are $E(h)$\textit{-local} objects, objects of the cofiber $\Spp/\Spp^{\le h}$ are \textit{complete away from} $E(h)$ and objects of $\Spp^{\ge h+1}$ are $E(h)$\textit{-acyclic}.
\end{definition}
The objects of this recollement fits into another recollement, which constructs the $K(h)$-local objects used in Morava K theory.
\begin{equation}
  \label{SppRecollement2}
    \begin{tikzcd}[sep = large]
    \Spp^{\ge h}/D(\mathbb{S}_p)^{\ge h+1} \arrow[dr, shift left=1ex, "i_\vee"] \arrow[dd, shift right=1.5ex, "i^\wedge i_\vee"'] \\
    {} & \Spp^{\le h}  \arrow[dl,shift  right=1ex, "i^\wedge" description]  \arrow[ul, shift left=1ex, "i^\vee"]  \arrow[r, bend left, "j^*"] \arrow[r, above, bend right, "j^\times"] & \arrow{l}{}[swap]{j_*} \Spp^{\le h-1} \\
    \Spp^{\le h}/\Spp^{\le h-1} \arrow[ur, shift right=1ex, "i_\wedge" description] \arrow[uu, shift right=1.5ex, "\sim", "i^\vee i_\wedge"']
    \end{tikzcd}
\end{equation}

\begin{definition}
In the recollement diagram (\ref{SppRecollement2}) we say that objects of $\Spp^{\ge h}/\Spp^{\ge h+1}$ are \textit{monochromatic of level h} and that objects of $\Spp^{\le h}/\Spp^{\le h-1}$ are $K(h)$\textit{-local}.
\end{definition}
From now on we work with the maps in the last diagram. Notice that the equivalence in the last recollement diagram looks like a rule of fractions if we write out the Verdier quotient $\Spp^{\le h} / \Spp^{\le h -1}$. In this diagram the unit $\operatorname{id} \to j_*j_*$ is $E(h)$\textit{-localisation} and we denote the component maps of the natural transforation as $X \mapsto L_{E(h)}X$, and similarly component maps of $\operatorname{id} \to i_\wedge i^\wedge$ are denoted as $X \mapsto L_{K(h)}X$. These are the Bousfield localisation functors with respect to $E(h)$ and $K(h)$. The Recollement square
\begin{center}
    \begin{tikzcd}
    \operatorname{id} \arrow{r} \arrow{d} & i_\wedge i^\wedge \arrow{d} \\
    j_*j^* \arrow{r} & j_*j^*i_\wedge i^\wedge
    \end{tikzcd}
\end{center}
is important in this case. Recall that this is a cartesian square, so we can recover $E(h)$-localisation $L_{E(h)}X$ of an object $X$ as a pullback
\begin{center}
    \begin{tikzcd}
    L_{E(h)}X \arrow{r} \arrow{d} & L_{K(h)}X \arrow{d} \\
    L_{E(h-1)}X \arrow{r} & L_{E(h-1)}L_{K(h)}X
    \end{tikzcd}
\end{center}
since $L_{E(h-1)}L_{E(h)}X = L_{E(h-1)}X$. Furthermore, such a cartesian square gives a long exact sequence of homotopy groups which can be quite useful. In (\cite{luriechromatichomotopytheorynotes}, Lecture 35) Lurie uses these facts to compute the $K(1)$ and $E(1)$-localisation of the sphere spectrum $\Sph$.

\section{Bousfield localisation of $p$-complete spectra}\label{Section:bousfieldlocal}

As mentioned we are interested in $E(h)$ and $K(h)$-localisations, and Bousfields article \cite{bousfield1979localization} concerns the case of $E(1)$-localisations, also known as $KU_p$-localisation for a prime $p$. The rest of this paper is devoted to calculating $E(1)$-localisations of $p$-complete spectra. Bousfield mentions two results in \cite{bousfield1979localization} which will be crucial to our calculations. These are the contents of following two theorems.
\begin{theorem}[Adams-Baird, Ravenel]\label{Adams-Baird, Ravenel}
  $E(1)$-localisation is smashing, i.e. $L_{E(1)}X \simeq X \otimes L_{E(1)} \mathbb{S}_p$. Equivalently, this says that $X \to X \otimes L_{E(1)} \mathbb{S}_p$ is an $E(1)$-localisation
\end{theorem}
In particular, Theorem \ref{Adams-Baird, Ravenel} says that we get a recollement square (\ref{SppRecollement2}) for $h=1$. The units of the $p$-adic integers $\Z_p^*$ acts on $E(1)$ by taking a unit $g \in \Z_p^*$ to the Adam's operation $\psi_g$, and this action turns out to very important to our calculations.
\begin{theorem}[Mahowald for $p=2$, Miller for $p$ odd]\label{Mahowald, Miller}
   The homotopy fix points of the action of $\Z_p^*$ on $E(1)$ satisfy that
     $$L_{E(1)}\mathbb{S}_p \simeq E(1)^{h\Z_p^*}.$$
\end{theorem}
Combining Theorem \ref{Adams-Baird, Ravenel} and \ref{Mahowald, Miller} lets us understand the $E(1)$-localisation of $p$-complete spectra simply by understanding the $E(1)$-localisation of the $p$-complete sphere spectrum $\mathbb{S}_p$. This is the contents of the rest of the article.

\subsection{$E(1)$-localisation of the $p$-complete sphere spectrum}

We understand a spectrum, by understanding its homotopy groups. In this section we will calculate the homotopy groups of $E(1)^{h\Z_p^*}$. We split the proof in two cases: where $p$ is odd and where $p=2$. Both proofs use the \textit{homotopy fixed points} spectral sequence, but require different approaches. In the course of our proofs, we outline some intermediate results in an effort of making these somewhat lengthy calculations easier to follow.

We do a quick recollection of some needed propertis of the $p$-adic integers. Recall that $\Z_p$ is defined as the inverse limit $\Z_p := \varprojlim_m \Z/p^m$ in the category of rings. Concretely, we can write a $p$-adic integer $x$ on the form
\[
  x=\sum_{n = 0}^\infty a_n p^n
\]
where $ 0 \le a_n \le p-1$ for all $n$.
An element $u$ of $\Z_p$ is a unit, if and only if, $u$ is not divisible by $p$, i.e. $a_0 \neq 0$.

\begin{remark}
  The homotopy groups of $E(1)$ are $\Z_p$ in the even degrees and zero otherwise, and we write $\pi_{2n}E(1) \simeq \Z_p(n)$ where $n$ is the called the \textit{weight}. Though the action of $\Z_p^*$ on $E(1)$ is complicated to describe, the induced action of $\Z_p^*$ on the homotopy groups $\pi_{2n}E(1)$ is simple. It is given by taking a $p$-adic unit $g$ and acting as multiplication by $g^n$.
\end{remark}

\subsubsection{$p$ odd}

We first treat the case where $p$ is and odd prime, which results in the following theorem
\begin{theorem}
 If $p$ is an odd prime then
 \[
  \pi_n E(1)^{h\Z_p^*} \simeq
    \begin{cases}
    \Z_p, & n=0,-1 \\
    \Z/p^{v_p(n)+1}\Z, & p-1 \text{ divides } n \\
    0, & \text{otherwise}
    \end{cases}
  \]
  where $v_p$ is $p$-adic valuation.
\end{theorem}
The first intermediate result will be that $p$ is an odd primee $\Z_p^*$ is topologically cyclic, which will be important in our calculations.
\begin{lemma}
  If $p$ is an odd prime, then the units of the $p$-adic integers $\Z_p^*$ is topologically cyclic.
\end{lemma}
\begin{proof}
The mod $p$ map $\Z_p^* \to \F_p^*$ is surjective, and has kernel $(1+p\Z_p)^*$ so there is a short exact sequence
\[
\begin{tikzcd}
1 \rar & (1+p\Z_p)^* \rar{} & \Z_p^* \rar{} & \F_p^* \rar & 1
\end{tikzcd}
\]
This sequence split since there is a section $\omega \colon \F_p^* \to \Z_p^*$. This is called the \textit{Teichmüller character}, and we can define it as follows. Let $\zeta \in \F_p^*$ be any $(p-1)$'st root of unity, and let $x \in \Z_p^*$ be any $p$-adic unit such that $x \equiv \zeta$ mod $p$. Set $\omega(\zeta) = \lim_{n \to \infty} x^{p^n}$. This converges in the $p$-adic topology and is independent of the choice of $x$,
and thus $\omega$ defines a multiplicative section. Hence
\[
\Z_p^* \cong (1 + p\Z_p)^* \times \F_p^*,
\]
and so $g = (1+p, \zeta)$ where $\zeta$ is any $(p-1)$'st root of unity, is a topological generator for $\Z_p^*$.
\end{proof}
In the remaining parts of this section, we will fix a topological generator $g \in \Z_p^*$. Our first intermediate result in the calculations is the following proposition.
\begin{proposition}
  If $p$ is an odd prime, then
  \[
   \pi_n E(1)^{h\Z_p^*} \simeq
     \begin{cases}
     \Z_p, & n=0,-1 \\
     \coker (g^n - \id), & -1 \neq n \text{ and $n$ odd} \\
     0, & \text{otherwise.}
     \end{cases}
   \]
   where $g^n \colon \Z_p(n) \to \Z_p(n)$ is multiplication with $g^n$.
\end{proposition}
\begin{proof}
There are different ways to show this proposition. We will use the \textit{homotopy fixed points spectral sequence}, which states that for a spectrum $X$ with a $G$-action for a group $G$, there is a spectral sequence with
\[
E^2_{i,j} = H^{-i}(BG, \pi_j(X)) \implies \pi_{i+j}(X^{hG}).
\]
In our case, $\Z_p^*$ acts on $E(1)$ and $\Z_p^*$ is topologically cyclic so the spectral sequence looks like
\[
E^2_{i,j} = H_c^{-i}(B\Z_p^*, \pi_j(E(1))) \implies \pi_{i+j}(E(1)^{h\Z_p^*}),
\]
where $H_c$ denotes continuous cohomology. Now using that the homotopy groups of $E(1)$ are concentrated in even degrees, we have that
\[
  H^i_c(B\Z_p^*, \pi_j(E(1))) =
\begin{cases}
\ker (g^j - \id), & i=0 \text{ and } j \text{ even} \\
\coker (g^j - \id), & i=1 \text{ and } j \text{ even} \\
0, & \text{otherwise.}
\end{cases}
\]
where $g^j \colon \Z_p(n) \to \Z_p(n)$ is multiplication by $g^j$. Looking at the second page of the spectral sequence we notice immediately that there are no non-trivial differentials, and in fact this is the case for any page $E^k$ for $k \ge 2$. So we can easily determine the homotopy groups. First of all
\begin{align*}
  \pi_0 E(1)^{h\Z_p^*} & \cong  H^0_c(B\Z_p^*, \pi_0 E(1)) \\
    & \cong \ker (\Z_p(0) \toWithMapLong{0} \Z_p(0)) \\
    & \cong \Z_p
\end{align*}
and similarly,
\begin{align*}
  \pi_{-1} E(1)^{h\Z_p^*} & \cong  H^1_c(B\Z_p^*, \pi_0 E(1)) \\
    & \cong \coker (\Z_p(0) \toWithMapLong{0} \Z_p(0)) \\
    & \cong \Z_p.
\end{align*}
When $j \neq 0$, $g^j - \id$ is an injection, and so $\ker(g^j - \id) = 0$. Hence
\begin{align*}
  \pi_{2j} E(1)^{h\Z_p^*} & \cong  H^0_c(B\Z_p^*, \pi_{2j} E(1)) \\
    & \cong \ker (\Z_p(j) \toWithMapLong{g^j - \id} \Z_p(j)) \\
    &  = 0
\end{align*}
Finally for $j \neq 0$, we have that
\begin{align*}
  \pi_{2j-1} E(1)^{h\Z_p^*}  \cong  H^1_c(B\Z_p^*, \pi_2j E(1))
    \cong \coker (\Z_p(j) \toWithMapLong{g^j - \id} \Z_p(j))
\end{align*}
which completes the proof of the proposition.
\end{proof}
What remains in our calculations is to identify the cokernel of $g^n-\id$ when $n \neq 0$. This requires some work with the $p$-adic integers, and will be the contents of our next lemma. To state the lemma we need the notion of the \textit{$p$-adic valuation} function $v_p \colon \Z_p \to \Z \cup \{\infty\}$. Every $p$-adic integer $x \in \Z_p$ can be uniquely written as $x = up^n$ where $u$ is a $p$-adic unit and $n$ is a non-negative integer. Set $v_p(x) = n$, with the convention that $v_p(0) = \infty$.
\begin{lemma}
  If $n$ is a positive integer then
  \[
    \coker(g^n - \id) \cong \Z/p^{v_p(n)+1}\Z
  \]
  when $p-1$ divides $n$ and zero otherwise.
\end{lemma}
\begin{proof}
We have a short exact sequence
\[
\begin{tikzcd}
0 \rar & \Z_p \rar{g^n - \operatorname{id}} & \Z_p \rar{} & \Z_p/{p^{v_p(g^n-1)}\Z} \rar & 0
\end{tikzcd}
\]
and so we study $\Z_p/{p^{v_p(g^n-1)}\Z}$, and in particular $v_p(g^n-1)$. As we saw earlier, there is a short exact sequence
\[
\begin{tikzcd}
1 \rar & (1+p\Z_p)^* \rar{} & \Z_p^* \rar{} & \F_p^* \rar & 1
\end{tikzcd}
\]
Now since $g$ is a $p$-adic unit, $p$ does not divide $g$ and so $g^n \equiv 1$ mod $p$ if, and only if, $p-1$ divides $n$.\todo{use lemma 2.12?} We split the proof in two cases.

First, consider the case where $p-1$ does not divide $n$. Then $g^n -1 \not\equiv 0$ mod $p$, which means that $g^n -1$ is a unit, so that $v_p(g^n - 1) = 0$. Hence
\[
\coker ( g^n - \id ) = 0.
\]
Now consider the case where $p-1$ divides $n$. This means that $g^n \equiv 1$ mod $p$, and so $g^n$ is in the kernel of the mod $p$ map, i.e. $g^n \in (1 + p\Z_p)^*$. We can then write $g^n = (g^{p-1})^{n/(p-1)}$ and since $g$ is a topological generator, so is $g^{p-1} \in (1 + p\Z_p)^*$. Since $p$ is odd, we have isomorphisms
\[
\begin{tikzcd}
  (1+ p\Z_p)^* \arrow[d, "\log"', "\sim", shift right=1.3ex] \\
  p\Z_p \arrow[u, "\exp"', shift right = 1.3ex]{exp}
\end{tikzcd}
\]
Hence the topological generator $g^{p-1}$ is sent to a topological generator of $p\Z_p$. Such a generator is of the form $pu$ where $u$ is a $p$-adic unit. Now $v_p(n) = v_p(\frac{n}{p-1})$ so we can write $\frac{n}{p-1} = tp^{v_p(n)}$ where $t$ is a $p$-adic unit. Then $\log(g^n) = \log(g^{p^{v_p(n)}(p-1)t}) = utp^{v_p(n)+1}$ and since $u$ and $t$ both are $p$-adic units, so is $ut$, hence $\log(g^n)$ generates the subgroup $p^{v_p(n)+1}\Z_p$.
Thus $g^n$ generates $(1 + p^{v_p(n)+1}\Z_p)^*$ and so $g^{n}-1$ generates the subgroup $p^{v_p(n)+1}\Z_p$ and we can write $g^n-1 =u' p^{v_p(n)+1}$ for a $p$-adic unit $u'$ so that
\iffalse
But if $g^{p-1} \mapsto pu$ then $g^n \mapsto \frac{n}{p-1}pu$ under $\log$, and $$v_p\left(\frac{n}{p-1}pu\right) = v_p(n)+1.$$ We conclude that \fi
\[
v_p(g^n - 1 ) = v_p(n) + 1
\]
which finishes the proof.
\end{proof}
Combining these two results finishes our calculations when $p$ is odd.

\subsubsection{$p=2$}

We show the following theorem

\begin{theorem}\label{theorem:p=2final}
  The homotopy groups of $E(1)^{h\Z_2^*}$ are given by canonical equivalences
  \[
  \pi_nE(1)^{h\Z_2^*} \simeq
    \begin{cases}
      \Z_2 (0), & n = -1 \\
      \Z/2  (1) \times \Z_2 (0), & n = 0 \\
      \Z / 2(\frac{n+2}{2}), & n \equiv 0 \text{ mod } 8 \text{ and } n \neq 0 \\
      \Z / 2 (\frac{n+2}{2}) \times \Z / 2 ( \frac{n+1}{2}), & n \equiv 1 \text{ mod } 8 \\
      \Z / 2 (\frac{n+2}{2}), & n \equiv 2 \text{ mod } 8 \\
      \Z / 8 (n/2), & n \equiv 3 \text{ mod } 8 \\
      \Z / 2^{v_2(n) + 1}\Z (n/2), & n \equiv 7 \text{ mod } 8 \text{ and } n \neq -1 \\
      0, & \text{otherwise.}
    \end{cases}
  \]
\end{theorem}
When $p=2$, then we cannot use the same argument to say that $\Z_2^*$ is topologically cyclic. Instead, notice that $\Z_2^* \cong (1 + 2\Z_2)^*$ and consider the short exact sequence
\[
\begin{tikzcd}
  1 \arrow{r} & (1 + 4\Z_2)^* \arrow{r} & (1+2\Z_2)^* \arrow{r} & \{\pm 1\} \rar & 1
\end{tikzcd}
\]
where the last non-trivial map is mod $4$. This splits since we can lift $-1$ to $1 + (-1)2$, and we obtain that
\[
\Z_2^* \cong (1 + 4\Z_2)^* \times \{\pm 1\}.
\]
We can now use the homotopy fixed spectral sequence as we did for the case where $p$ is odd. It follows from a general fact about group actions that
\[
E(1)^{h\Z_2^*} \cong \left(E(1)^{h\{\pm 1\}}\right)^{h(1 + 4\Z_2)^*}
\]
so we start by calculating $E(1)^{h\{\pm 1\}}$ using the spectral sequence.

\begin{proposition}\label{proposition:homotopypm1}
  The non-trivial homotopy groups of the homotopy fixed points of $\{\pm 1\}$ on $E(1)$ are given by
  \[
    \pi_iE(1)^{h\{\pm 1\}} \cong
    \begin{cases}
      \Z_2 (i /2), & i \equiv 0 \text{ mod } 8 \\
      \Z/2 (\frac{i+1}{2}), & i \equiv 1 \text{ mod } 8 \\
      \Z/2 (\frac{i+2}{2}), & i \equiv 2 \text{ mod } 8 \\
      2 \Z_2 (i/2), & i \equiv 4 \text{ mod } 8
    \end{cases}
  \]
\end{proposition}
\begin{proof}
As mentioned we consider the homotopy fixed point spectral sequence, gives us a spectral sequence such that
\[
E^2_{i,j} = H^{-i}(B\{\pm 1\}), \pi_j E(1)) \implies \pi_{i+j}E(1)^{h \{\pm 1\}}.
\]
We calculate the group cohomology $H^{i}(B\{\pm 1\}), \pi_j E(1))$ when $j$ is even, otherwise it is trivial. The group cohomology is calcaulated as the cohomology of the complexes
\begin{equation}\label{equation:groupcohomcomplex}
  \begin{tikzcd}
    \cdots & \Z_2(n) \lar[swap]{\psi_{-1} + 1} & \Z_2(n) \lar[swap]{\psi_{-1} - 1}  & \Z_2 (n) \lar[swap]{\psi_{-1} + 1}  & \Z_2 (n) \lar[swap]{\psi_{-1} - 1}
  \end{tikzcd}
\end{equation}
If $n = j/2$ is an even integer then $\psi_{-1} = (-1)^{n} = 1$, so the complex (\ref{equation:groupcohomcomplex}) actually looks like
\[
\begin{tikzcd}
  \cdots & \Z_2(n) \lar[swap]{2} & \Z_2(n) \lar[swap]{0}  & \Z_2 (n) \lar[swap]{2}  & \Z_2 (n) \lar[swap]{0}
\end{tikzcd}
\]
When $n$ is odd $\psi_{-1} = (-1)^n = -1$, so the complex (\ref{equation:groupcohomcomplex}) is
\[
\begin{tikzcd}
  \cdots & \Z_2(n) \lar[swap]{0} & \Z_2(n) \lar[swap]{-2}  & \Z_2 (n) \lar[swap]{0}  & \Z_2 (n) \lar[swap]{-2}
\end{tikzcd}
\]
Since the two maps
\[
\begin{tikzcd}
  \Z_2(n)  \arrow[r, "2", shift left]  \arrow[r, "-2", shift right, swap] & \Z_2(n)
\end{tikzcd}
\]
are injections, the kernel is trivial and their cokernel is $\Z/2$. Using this to calculate the cohomology of our two chain complexes one obtains the following $E^2$-page of the spectral sequence.
\begin{table}[H]
  \centering
  \setlength{\tabcolsep}{12pt}
  \setlength{\extrarowheight}{2pt}
\begin{tabular}{ccccc|c}
  $\ddots$      & $\vdots$ &   $\vdots$ &  $\vdots$  & $\vdots$  & $\vdots$ \\
  $\cdots$      & $\Z/2(3)$ &   0 &  $\Z/2(3)$  & 0  & 6 \\
  $\cdots$      & 0 &   0 &  $0$  & $0$  & 5 \\
    $\cdots$     & 0  &   $\Z/2(2)$ &  0  & $\Z_2(2)$  & 4 \\
    $\cdots$     &  0 &   0 &  0  &  0 & 3 \\
    $\cdots$     & $\Z/2(1)$  &  0  &  $\Z/2(1)$  &  0 & 2 \\
    $\cdots$     & 0 &  0  &  0  & 0  & 1 \\
    $\cdots$     & 0 &  $\Z/2(0)$  & 0 &  $\Z_2(0)$  & 0 \\ \hline
$\cdots$  & -3 & -2 & -1 & 0 &
\end{tabular}
\end{table}
\noindent Notice that there are no non-trivial differentials on page two and so $E^2 = E^3$ but on the $E^3$ page there are possible non-trivial differentials. Let $u \in H^0(B\{\pm 1\}, \pi_4E(1)) \cong \Z_2(2)$ and $t \in H^1(B\{\pm 1\}, \pi_2E(1)) \cong \Z/2$ denote generators. The generators are then placed as seen in the following diagram.
\begin{table}[H]
  \centering
  \setlength{\tabcolsep}{8pt}
  \setlength{\extrarowheight}{2pt}
\begin{tabular}{ccccc|c}
  $\ddots$      & $\vdots$ &   $\vdots$ &  $\vdots$  & $\vdots$  & $\vdots$ \\
  $\cdots$      & $t^3$ &   0 &  $tu$  & 0  & 6 \\
  $\cdots$      & 0 &   0 &  $0$  & $0$  & 5 \\
    $\cdots$     & 0  &   $t^2$ &  0  & $u$  & 4 \\
    $\cdots$     &  0 &   0 &  0  &  0 & 3 \\
    $\cdots$     & $t^3u^{-1}$  &  0  &  $t$  &  0 & 2 \\
    $\cdots$     & 0 &  0  &  0  & 0  & 1 \\
    $\cdots$     & 0 &  $t^2u^{-1}$  & 0 &  $1$  & 0 \\ \hline
$\cdots$  & -3 & -2 & -1 & 0 &
\end{tabular}
\end{table}
\noindent Notice that $2t=0$
and since we have non-trivial homotopy groups of $E(1)$
in the even negative degrees, $u$ is invertible, and so
$E^2$ is the algebra $\Z_2[t, u^{\pm 1}]/(2t)$.

We now look at the differentials on the $E^3$-page, and as it turns out, most of them disappear. First notice that $E(1)$ is an $E_\infty$-ring spectrum, and so there is a map $\Sph \to E(1)$ since $\Sph$ is initial among $E_\infty$-ring spectra. It is a non-trivial fact that on homotopy groups this maps sends the Hopf map $\eta \in \pi_1(\Sph)$ to our generator $t$, and since we know that $\eta^4 \in \pi_4(\Sph) = 0$ we must have that $\eta^4 = 0$ and so $t^4 = 0$. Hence the differential on $E^3$ satisfy that $0 = d^3 t^4$
which then forces $d^3t = 0$ by the multiplicative structure on the differentials.

 We now claim that $d^3(u) = t^3$. To see this consider the following short exact sequence of bigraded abelian groups
 \[
\begin{tikzcd}
  0 \rar & 2\Z_2[u^{\pm 1}] \rar & \Z_2[t, u^{\pm 1}]/(2t) \rar & \F_2[t, u^{\pm 1}] \rar & 0.
\end{tikzcd}
 \]
 For degree reasons $d^3 \colon \Z_2[t, u^{\pm 1}]/(2t) \to \Z_2[t, u^{\pm 1}]/(2t)$ is zero on $2\Z_2[u^{\pm 1}]$, so it induces a map on the exact sequence above to itself. In particular this means that $d^3$ is determined by the induced map on $\F_2[t, u^{\pm 1}]$. In $\F_2[t, u^{\pm 1}]$ we have that $0 = t^4 = d^3(tu) = t d^3(u)$ and since
 $t$ is not a zero-divisor in $\F_2[t, u^{\pm 1}]$,
 the induced differential sends $u$ to $t^3$. This forces $d^3(u) = t^3$ as well, which proves our claim.


Since we now know the effect of $d^3$ on both $u$ and $t$ the multiplicative structure of the differentials then tells us how all the differentials on the $E^3$-page looks. \todo{finish} \iffalse
\begin{table}[H]
  \centering
  \setlength{\tabcolsep}{12pt}
  \setlength{\extrarowheight}{2pt}
\begin{tabular}{ccccc|c}
  $\ddots$      & $\vdots$ &   $\vdots$ &  $\vdots$  & $\vdots$  & $\vdots$ \\
  $\cdots$      & 0 &   0 &  $\Z/2(3)$  & 0  & 6 \\
  $\cdots$      & 0 &   0 &  $0$  & $0$  & 5 \\
    $\cdots$     & 0  &   $\Z/2(2)$ &  0  & $\ker(d^3)$  & 4 \\
    $\cdots$     &  0 &   0 &  0  &  0 & 3 \\
    $\cdots$     & 0 &  0  &  $\Z/2(1)$  &  0 & 2 \\
    $\cdots$     & 0 &  0  &  0  & 0  & 1 \\
    $\cdots$     & 0 &  $\Z/2(0)$  & 0 &  $\Z_2(0)$  & 0 \\ \hline
$\cdots$  & -3 & -2 & -1 & 0 &
\end{tabular}
\end{table}
\noindent \fi In particular are no non-trivial differentials on $E^k$ for $k \ge 4$ so $E^4 = E^\infty$, and we can calculate the desired homotopy group from the differentials on the page $E^3$. We notice that the kernel of the surjective map
\[
d^3 \colon \Z_2(2) \to \Z/2(3)
\]
is generated by $2u$ since $d^3(2u)=2t^3 = 0$, and the domain $\Z_2(2)$ sits in total degree $4$. Using the multiplicative structure of the differentials, this is the case in any total degree $n$ where $n \equiv 4$ mod $8$, and we arrive at the homotopy groups we desire. \todo{elaborate a bit why everything vanishes}
\end{proof}

\begin{remark}
  Some may recognize these homotopy groups as the homotopy groups of the real K-theory spectrum $KO$, and it is in fact true that
  \[
    E(1)^{h\{ \pm1\}} \simeq KO.
  \]
\end{remark}

For the last part of the calculations we will need a small lemma about the $2$-adic valuation.

\begin{lemma}\label{lemma:2adicval}
  The $2$-adic valuation satisfies that $v_2(1-5^n) = 2 + v_2(n)$ when $n$ is even.
\end{lemma}
\begin{proof}
This is a direct consequence of case ii) in (\cite{adams}, Lemma 2.12)
\end{proof}

\begin{proof}[Proof of Theorem \ref{theorem:p=2final}]
  What remains to be calculated is thus $\left(E(1)^{h\{\pm 1\}}\right)^{h(1 + 4\Z_2)^*}$. Again we use the homotopy fixed point spectral sequence, which in this case yields
  \[
  E^2_{i,j} = H^{-i}(B(1+4 \Z_2)^*, \pi_jE(1)^{h\{\pm 1\}}) \implies \pi_{i+j}\left(E(1)^{h\{\pm 1\}}\right)^{h(1 + 4\Z_2)^*}
  \]
  Notice that the logarithm gives an equivalence $(1+4\Z_2)^*\toWithMapLong{\simeq} 4\Z_2$ and multiplication by $4$ gives an equivalence $\Z_2 \toWithMapLong{\simeq} 4\Z_2$, so it suffices to calculate the group cohomology of $\Z_2$ with coefficients in $\pi_jE(1)^{h\{\pm 1\}}$. Using that $5$ generates $(1+4\Z_2)^*$ calculations yields that
  $H^i(\Z_2, \pi_jE(1)^{h\{\pm 1\}})$ vanishes for $i \ge 2$, and for $i =0,1$ the cohomology group depends on the weight of the coefficients.
  \iffalse
  \begin{center}
    \begin{tikzpicture}
    \matrix (m) [matrix of math nodes,
      nodes in empty cells,nodes={minimum width=5ex,
      minimum height=5ex,outer sep=0pt},
      column sep=1ex,row sep=1ex]{
        \cdots & &      &     &     & \\
           \cdots &    &  \Z &  0  & \Z & \\
           \cdots  &   &  \Z  & \Z &  \Z_2(0 )  & \\
      \strut\cdots  &   -2  &  -1  & 0 & \strut \\};
  \draw[thick] (m-1-6.west) -- (m-4-6.west) ;
  \draw[thick] (m-4-1.north) -- (m-4-6.north) ;
  \end{tikzpicture}
  \end{center}
  \fi
   Using Proposition \ref{proposition:homotopypm1} calculations of the group cohomology of $\Z_2$ then comes down to calculating the kernel and cokernel of the maps
  \begin{enumerate}[i)]
    \item $\Z_2(i/2) \toWithMapLong{1-5^{i/2}} \Z_2(i/2)$ when $i \equiv 0$ mod $8$
    \item $\Z/2(\frac{i+1}{2}) \toWithMapLong{1-5^{(i+1)/2}} \Z/2(\frac{i+1}{2})$ when $i \equiv 1$ mod $8$
    \item $\Z/2(\frac{i+2}{2}) \toWithMapLong{1-5^{(i+2)/2}} \Z/2(\frac{i+2}{2})$ when $i \equiv 2$ mod $8$
    \item $2\Z_2(i/2) \toWithMapLong{1-5^{i/2}} 2\Z_2(i/2)$ when $i \equiv 4$ mod $8$
  \end{enumerate}
  We do this each case at a time.
  \begin{enumerate}[i)]
    \item When $i=0$, $1-5^0 = 0$ so it has kernel and cokernel $\Z_2(0)$. When $i \equiv 0$ mod $8$ and $i \neq 0$, $1-5^{1/2}$ the map is injective and the cokernel is given by $\Z/2^{v_2(1-5^{i/2})}\Z$. By Lemma \ref{lemma:2adicval} we have that $v_2(1-t^{i/2}) = 2 + v_2(i/2) = 1 + v_2(i)$,
    which shows that the cokernel is $\Z/2^{v_2(i)+1}\Z$.
    \item In this case $1-5^{(i+1)/2}$ will be even, so it will actually be the zero map.
    \item The same is true here. It will be the zero map.
    \item This case requires a bit of work. We have seen that the cokernel of $1-t^{i/2}$ is $\Z/2^{v_2(1-t^{i/2})}\Z$, and so we calculate $v_2({1-t^{i/2}})$ when $i \equiv 4$ mod $8$. $i \equiv 4$ mod $8$ implies that $i/2 \equiv 2$ or $i/2 \equiv 6$ mod $8$ so that $v_2(i/2) = 1$
    and from Lemma \ref{lemma:2adicval} we get that $v_2(1-t^{i/2}) = 2 + v_2(i/2) = 3$. Thus $1-t^{i/2}$ is an injective map with cokernel $\Z/2^3 \cong \Z/8$
  \end{enumerate}
  From this we see that the $E^2$-page is $8$-periodic and will look like the following
  \begin{table}[H]
    \centering
    \setlength{\tabcolsep}{12pt}
    \setlength{\extrarowheight}{2pt}
  \begin{tabular}{cccc|c}
    $\ddots$        &   $\vdots$ &  $\vdots$  & $\vdots$  & $\vdots$ \\
    $\cdots$        &   0 &  $\Z/2(6)$  & $\Z/2(6)$  & 10 \\
    $\cdots$        &   0 &  $\Z/2(5)$  & $\Z/2(5)$  & 9 \\
    $\cdots$        &   0 &  $\Z/2^{v_2(8)+1}\Z(4)$  & $0$  & 8 \\
    $\cdots$        &   0 &  $0$  & $0$  & 7 \\
    $\cdots$        &   0 &  $0$  & $0$  & 6 \\
    $\cdots$        &   0 &  $0$  & $0$  & 5 \\
      $\cdots$        &   0 &  $\Z/8 (2)$  & $0$  & 4 \\
      $\cdots$        &   0 &  0  &  0 & 3 \\
      $\cdots$        &  0  &  $\Z/2(2)$  &  $\Z/2(2)$ & 2 \\
      $\cdots$       &  0  &  $\Z/2(1)$  & $\Z/2(1)$  & 1 \\
      $\cdots$       &  0  & $\Z_2(0)$ &  $\Z_2(0)$  & 0 \\ \hline
  $\cdots$  & -2 & -1 & 0 &
  \end{tabular}
  \end{table}
  \noindent There are no non-trivial differentials on page $E^k$ for $k \ge 2$, so this page is actually the $E^\infty$ page, and so most of the homotopy groups we can read directly from this page. There are a few cases where we get an extension problem. These are in total degree $0$ and $n$ when $n \equiv 1$ mod $8$. In the first case we have a short exact sequence
  \[
  \begin{tikzcd}
    0 \rar & \Z/2(1) \rar & \pi_0E(1)^{h\Z_2^*} \rar & \Z_2(0) \rar & 0
  \end{tikzcd}
  \]
  which splits since $\Z_2(0)$ is a free, hence projective, $\Z_2$-module. Thus
  \[
  \pi_0E(1)^{h\Z_2^*} \cong \Z/2(1) \times \Z_2(0),
  \]
  but there are multible ways to construct this splitting, and in fact we can choose a canonical way. We have a unique map of $E_\infty$-ring spectra $\Sph \to E(1)$ as we have seen earlier, which induces a map on homotopy groups
  \[
  \pi_n \Sph_2 \to \pi_nE(1)^{h\Z_2^*}
  \]
  where $\Sph_2$ is the $2$-local sphere spectrum. On $\pi_0$ we have an equivalence
  \[
  \pi_0\Sph_2 \toWithMapLong{\simeq} \Z_2(0)
  \]
  and we can choose the section $\Z_2(0) \to \pi_0 E(1)^{h\Z_2^*}$ such that the diagram
  \[
  \begin{tikzcd}
    \pi_0E(1)^{h\Z_2^*} & \Z_2(0) \lar \\
    \pi_0 \Sph_2 \arrow{u} \arrow{ur} &
  \end{tikzcd}
  \]
  commutes. This gives us the cannonical splitting
  \[
  \Z/2(1) \times \Z_2(0) \toWithMapLong{\simeq} \pi_0E(1)^{h\Z_2^*}.
  \]
  When $n \equiv 1$ mod $8$, we can do a similar thing. We have a short exact sequence
  \[
  \begin{tikzcd}
    0 \rar & \Z/2(\frac{n+2}{2}) \rar & \pi_nE(1)^{h\Z_2^*} \rar & \Z/2(\frac{n+1}{2}) \rar & 0
  \end{tikzcd}
  \]
  hence $\pi_nE(1)^{h\Z_2^*} \cong \Z/2(\frac{n+2}{2}) \times \Z/2 (\frac{n+1}{2})$. But there are two ways to construct a section $\Z/2(\frac{n+1}{2}) \to \pi_nE(1)^{h\Z_2^*}$. Again choose the one commuting
  \[
  \begin{tikzcd}
    \pi_nE(1)^{h\Z_2^*} & \Z/2(\frac{n+1}{2}) \lar \\
    \pi_n \Sph_2 \arrow{u} \arrow{ur} &
  \end{tikzcd}
  \]
  to get a cannonical splitting
  \[
  \begin{tikzcd}
    \Z/2(\frac{n+2}{2}) \times \Z/2(\frac{n+1}{2}) \rar{\simeq} & \pi_nE(1)^{h\Z_2^*}.
  \end{tikzcd}
  \]
  This concludes our proof.
\end{proof}

\begin{thebibliography}{9}

\bibitem{bousfield1979localization}
  Aldridge K.Bousfield,
  \textit{The localization of spectra with respect to homology},
  1979.

\bibitem{adams}
  J. F. Adams,
  \textit{On the groups J(X)-II},
  1965.

  \bibitem{barwickglasman}
    Clark Barwick and Saul Glasman,
    \textit{A note on stable recollements},
    2016.

\bibitem{luriechromatichomotopytheorynotes}
  Jacob Lurie,
  \textit{Chromatic homotopy theory lecture notes},
  2010.

\end{thebibliography}

\end{document}
